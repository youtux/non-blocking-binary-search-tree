\documentclass[a4paper,draft,12pt]{article}

\usepackage{times}
\usepackage{url}
\usepackage{hyperref}
\usepackage{float}
\usepackage{acronym}
\usepackage{paralist}
\usepackage{graphicx}
\usepackage{listings}
\usepackage{acronym}
\usepackage{pdfpages}

% \acrodef{EJB}{Enterprise Java Bean}
\acrodef{CAS}{compare-and-swap}

\lstset{ %
    breaklines=true,
    basicstyle=\footnotesize,
    moredelim=[is][\bfseries]{[*}{*]}
}

% ----------------------------------------------------------------------------
% Start the document
%

\title{\textbf{A Java implementation of Non-blocking Binary Search Trees}} % Article title

\author{
    \textsc{Alessio Bogon}\\[2mm] % Your name
    \normalsize University of Trento - Master Degree in Computer Science\\ % Your institution
    \normalsize Concurrency\\
    \href{mailto:alessio.bogon@studenti.unitn.it}{alessio.bogon@studenti.unitn.it} % Your email address
}
\date{\today}

\begin{document}

\maketitle

% - In REPORT describe the main difficulties you encountered, how you solved
% them, and explain informally why you believe it works

\section{Introduction} %(fold)
\label{sec:introduction}
This project is a non-blocking concurrent Java implementation of the binary search tree structure, and aims to reproduce the work of Ellen et al\cite{ellen10}.
The basic interface that we target is the \texttt{Set<T>}, with the following operations: \texttt{find (T item)}, \texttt{insert(T item)} and \texttt{remove(T item)}.
The class that implements it is \texttt{Non\-Blocking\-Binary\-Search\-Tree\-<T>}.

All the implemented methods are total and lock-free, but the \texttt{find} method is also wait-free.
This is due to the fact that \texttt{insert} and \texttt{delete} operations might need to adjust the tree in order to complete, while \texttt{find} can just look for the item without \emph{helping} other concurrent operations.


% section introduction (end)

\section{Usage} % (fold)
\label{sec:usage}
The program can be executed by running the provided script on UNIX platforms:
\begin{verbatim}
    ./run.sh
\end{verbatim}
It basically compiles all the sources needed by the main class and then run it.
Please ensure that the \texttt{java} executable is available in your \texttt{PATH}.
You can also configure the verbosity level of the output inside the script, which allows to trace the various \ac{CAS} operations.
The default level is \emph{FINE}, which shows the methods invocations and return values of the interface methods.

We report a simple execution where Thread1 inserts
\begin{verbatim}
[...] Thread#13:    insert: ENTRY 10
[...]
[...] Thread#13:    insert: RETURN true
[...] Thread#13:    insert: ENTRY 20
[...]
[...] Thread#13:    insert: RETURN true
[...] Thread#13:    insert: ENTRY 30
[...]
[...] Thread#13:    insert: RETURN true
[...] Thread#14:      delete: ENTRY 20
[...]
[...] Thread#14:      delete: dflag[key=20] success
[...]
[...] Thread#15:         insert: ENTRY 15
[...]
[...] Thread#15:         helpDelete: mark[key=30] success
[...]
[...] Thread#15:         helpMarked: dchild[key=20]
[...] Thread#15:         helpMarked: dunflag[key=20]
[...] Thread#15:         helpMarked: RETURN
[...] Thread#15:         helpDelete: RETURN true
[...] Thread#15:         help: RETURN
[...]
[...] Thread#15:         insert: iflag[key=20] success
[...]
[...] Thread#15:         helpInsert: ichild[key=20]
[...] Thread#15:         helpInsert: iunflag[key=20]
[...] Thread#15:         helpInsert: RETURN
[...] Thread#15:         insert: RETURN true
[...] Thread#14:      helpDelete: mark[key=30] done by another thread
[...]
[...] Thread#14:      helpMarked: dchild[key=20]
[...] Thread#14:      helpMarked: dunflag[key=20]
[...] Thread#14:      helpMarked: RETURN
[...] Thread#14:      helpDelete: RETURN true
[...] Thread#14:      delete: RETURN true
[...]
\end{verbatim}




\section{Architecture} % (fold)
\label{sec:architecture}

\paragraph{CAS semantics}
In order to achieve the \emph{non-blocking} feature we use the Java variant of the \ac{CAS} operation, which is the compare-and-set.
The key difference is that compare-and-swap returns the value the object had before the operation, while compare-and-set returns \emph{true} if and only if the operation has been completed successfully.
For this reason, the Java implementation is not a 1:1 rewriting of the code given by the reference paper, and some trickery is needed when dealing with the compare-and-set results.

\section{Tests} % (fold)
\label{sec:tests}

% section tests (end)


\begin{thebibliography}{9}
    \bibitem{ellen10} Ellen, F., Fatourou, P., Ruppert, E., and van Breugel, F. \emph{Non-blocking Binary Search Trees}. Proceedings of the 29th Annual ACM Symposium on Principles of Distributed Computing (PODC), 131–140, 2010.
\end{thebibliography}


\end{document}
