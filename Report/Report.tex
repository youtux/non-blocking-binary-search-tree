\documentclass[a4paper,draft,12pt]{article}

\usepackage{times}
\usepackage{url}
\usepackage{hyperref}
\usepackage{float}
\usepackage{acronym}
\usepackage{paralist}
\usepackage{graphicx}
\usepackage{listings}
\usepackage{acronym}
\usepackage{pdfpages}
\usepackage{mdwlist}
\usepackage[margin=2cm]{geometry}

% \acrodef{EJB}{Enterprise Java Bean}
\acrodef{CAS}{compare-and-swap}

\lstset{ %
    breaklines=true,
    basicstyle=\footnotesize,
    moredelim=[is][\bfseries]{[*}{*]}
}

% ----------------------------------------------------------------------------
% Start the document
%

\title{\textbf{A Java implementation of \\Non-blocking Binary Search Trees}} % Article title

\author{
    \textsc{Alessio Bogon}\\[2mm] % Your name
    \normalsize University of Trento - Master Degree in Computer Science\\ % Your institution
    \normalsize Concurrency\\
    \href{mailto:alessio.bogon@studenti.unitn.it}{alessio.bogon@studenti.unitn.it} % Your email address
}
\date{\today}

\begin{document}

\maketitle

% - In REPORT describe the main difficulties you encountered, how you solved
% them, and explain informally why you believe it works

\section{Introduction} %(fold)
\label{sec:introduction}
This project is a non-blocking concurrent Java implementation of the binary search tree structure, and aims to reproduce the work of Ellen et al\cite{ellen10}.
The basic interface that we target is the \texttt{Set<T>}, with the following operations: \texttt{find (T item)}, \texttt{insert(T item)} and \texttt{remove(T item)}.
The class that implements it is \texttt{Non\-Blocking\-Binary\-Search\-Tree\-<T>}.

All the implemented methods are total and lock-free, but the \texttt{find} method is also wait-free.
This is due to the fact that \texttt{insert} and \texttt{delete} operations might need to adjust the tree in order to complete, while \texttt{find} can just look for the item without \emph{helping} other concurrent operations.

The project also provides the following features:
\begin{itemize*}
    \item \textbf{Java ``generics''} When useful, classes are declared using generic types.
    \item \textbf{Javadoc} Classes documentation.
    \item \textbf{OCaml-like output syntax} Tree representation in OCaml-like style.
    \item \textbf{Dot output syntax} Tree representation as a dot file.
    \item \textbf{Logging facility} Fine-grained logging facility: you can choose to print just method invocations/\-returns, or get deeper and show also the result of CAS operations.
\end{itemize*}



% section introduction (end)

\section{Usage} % (fold)
\label{sec:usage}
The program can be executed by running the provided script on UNIX platforms:
\begin{verbatim}
    ./run.sh
\end{verbatim}
It basically compiles all the sources needed by the main class and then run it.
Please ensure that you are using Java 7 or above and that the \texttt{java} and \texttt{javac} executables are available in your \texttt{PATH}.
You can also configure the verbosity level of the output inside the script, which allows to trace the various \ac{CAS} operations.
The default level is \emph{FINE}, which shows the methods invocations and return values of the interface methods.
The main class will also save the dot representation of the resulting graph inside the \texttt{graph.dot} file.
Please refer to \texttt{README.txt} for further information.

The provided Main class uses five threads and executes many operations, but for the sake of brevity we report this simple history, where three threads tries to do the following operations:

\begin{table}[H]
\centering

\label{my-label}
\begin{tabular}{c|c|c}
{\bf Thread\#10} & {\bf Thread\#11} & {\bf Thread\#12} \\
\hline
\texttt{bst.insert(20)}   & \texttt{bst.insert(15)}   & \texttt{bst.insert(30)}   \\
\texttt{bst.insert(30)}   & \texttt{bst.delete(20)}   & \texttt{bst.find(20)}     \\
                 & \texttt{bst.find(15)}     & \texttt{bst.insert(15)}   \\
                 &                  & \texttt{bst.find(20)}
\end{tabular}
\caption{My caption}
\end{table}

One possible execution is the following:
\begin{verbatim}
[02:03:55.648] Thread#10:    insert: ENTRY 20
[02:03:55.648] Thread#12:        insert: ENTRY 30
[02:03:55.648] Thread#11:      insert: ENTRY 15
[02:03:55.652] Thread#10:    insert: RETURN true
[02:03:55.653] Thread#10:    insert: ENTRY 30
[02:03:55.653] Thread#10:    insert: RETURN true
[02:03:55.654] Thread#11:      insert: RETURN true
[02:03:55.654] Thread#11:      delete: ENTRY 20
[02:03:55.655] Thread#12:        insert: RETURN false
[02:03:55.655] Thread#12:        find: ENTRY 20
[02:03:55.655] Thread#12:        find: RETURN true
[02:03:55.655] Thread#12:        insert: ENTRY 15
[02:03:55.656] Thread#12:        insert: RETURN false
[02:03:55.656] Thread#11:      delete: RETURN true
[02:03:55.656] Thread#12:        find: ENTRY 20
[02:03:55.656] Thread#11:      find: ENTRY 15
[02:03:55.657] Thread#12:        find: RETURN false
[02:03:55.657] Thread#11:      find: RETURN true
\end{verbatim}

As you can verify, this is a valid execution: Thread 10 and 11 successfully complete all the operations. Thread 12 fails to insert 30 because it already exists, then it finds 20 because Thread 10 inserted it before (and Thread 11 didn't delete it yet), then it fails again trying to insert 15 and then it looks for 20 and can't find it since Thread 11 managed to delete it.

\section{Intuition}
% TODO

\section{Architecture} % (fold)
\label{sec:architecture}
% TODO: tosay
% javadoc
% cas semantics
% atomic reference
% generics
% dot format
% ocaml format
% logging facility (thread safe?)
The provided implementation is basically a rewriting of the work done by Ellen et al in Java. Some trickery is needed when dealing with the \ac{CAS} objects.

\paragraph{CAS objects}
In order to achieve the \emph{non-blocking} feature we use the Java variant of the \ac{CAS} operation, which is the compare-and-set.
The key difference is that compare-and-swap returns the value the object had before the operation, while compare-and-set returns \emph{true} if and only if the operation has been completed successfully.

In the paper, the result of a compare-and-swap operation is checked by comparing the value returned with the one we are expecting. If they are equals, this means that the operation was successful, and viceversa.
With the compare-and-set semantics, it is sufficient to check the return value.

The main difficulty comes when we need to check the reason why an operation was unsuccessful.
This happens in the \texttt{helpDelete()} method: it tries to \emph{MARK} the parent node of the given operation, but the compare-and-set might fail because some other thread already marked it.
To solve this, the result of the \ac{CAS} operation is checked. If it failed, the current state of the parent is obtained and checked if it has been \emph{MARK}ed by another node.
This works fine because, intuitively, once a node has been marked, it cannot change its state (from this state, the \texttt{delete()} operation can only succeed) and will eventually be physically deleted.

% TODO
Another problem arises when trying to implement the CAS word

\section{Tests} % (fold)
\label{sec:tests}

% section tests (end)


\begin{thebibliography}{9}
    \bibitem{ellen10} Ellen, F., Fatourou, P., Ruppert, E., and van Breugel, F. \emph{Non-blocking Binary Search Trees}. Proceedings of the 29th Annual ACM Symposium on Principles of Distributed Computing (PODC), 131–140, 2010.
\end{thebibliography}


\end{document}
